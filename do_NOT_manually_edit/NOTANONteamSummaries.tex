%______________________________________
% Instructions and LaTeX code to copy/paste into your document, to incorporate this file into your proposal, now located at:
%
% https://github.com/pmarcum/WorPT-Work-Plan-Tool-4-proposals/blob/main/latexTableTemplates/NOTANONteamSummaries/readme.md
%______________________________________
\def\WorPTfolder{NOTANONteamSummaries} % set the folder name where WorPT files are contained
% ___________________________________________
%       Default formatting preferences
% ___________________________________________
%--- boldfaces the name at the beginning of each paragraph associated with that name
\def\NameFontstyle#1{\textbf{#1}}            % boldface the name at top of paragraph for that person
\newenvironment{NOTANONteamSummaries}{%
Mr. \NameFontstyle{Pablo M. Sanchez Alarcon}, \NameFontstyle{Science PI}, will lead the identification of fields and the second publication, a paper on galaxy identification.  Her background in extragalactic astronomy is essential for successful implementation of these tasks. In addition to these responsibilities, she will provide expertise in the model preparation by generating simulated input data, and in the incorporation of thermal emission models.  She will also assist with documentation and will co-author the first publication. \par
Dr. \NameFontstyle{Pamela M. Marcum}, \NameFontstyle{Administrative PI}, will lead the emission maps production, and the reduction of image mosaics related to archive applications. Finally, he will lead the first publication, a paper describing the pipeline and improved images.  His $\sim$15 years of experience in image analysis are needed for successful and timely completion of these tasks. In addition to leading these tasks, he will assist with model preparation by generating simualted data, and will identify fields of interest for the archive-related work, as well as help with code documentation and the development of the second publication. \par
Dr. \NameFontstyle{Alejandro S. Borlaff}, \NameFontstyle{co-Investigator}, will lead simulated data generation for models input and the incorporation of thermal models, as well as code documentation.  Her extensive work with model simulations and archival processes are well-matched to these roles.  Additionally, her expertise will be used to assist with generation of emission maps, identifying fields of interest, reduction of image mosaics and in the development of both publications. \par
Prof. \NameFontstyle{Sebastien  Comeron}, \NameFontstyle{co-Investigator},  \par
Prof. \NameFontstyle{Johan  Knapen}, \NameFontstyle{co-Investigator}, will assist with fields and galaxy identification and image mosaic reduction, as well as with the development of Papers~1 and 2. Her extensive experience in image analysis will significantly reduce the risk of false positive detections created by artifacts. \par
Dr. \NameFontstyle{Reynier  Peletier}, \NameFontstyle{co-Investigator},  \par
Prof. \NameFontstyle{Javier  Roman}, \NameFontstyle{co-Investigator},  
}
{%
}
